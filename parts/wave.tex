% vim: ts=2 sw=2 et tw=78:

\section{1D Wave Equation}

The homogeneous one dimensional wave equation is a linear second order
hyperbolic PDE:
\[
  u_{tt} - c^2 u_{xx} = 0
  \qquad (x, t) \in \mathbb{R} \times (0, \infty),
\]
where $c$ is to be interpreted as being a ``velocity''. To get an algebraic
intuition for the solution we can rewrite it as
\[
  (\partial_t - c\partial_x)(\partial_t + c\partial_x) u = 0,
\]
from which we can recognize the terms as transport equations and conclude that
the solutione are a linear combination of forward or backwards traveling
waves. The general solution to this problem can thus be written as
\begin{equation} \label{eqn:wave:solution}
  u(x,t) = p(\xi) + q(\eta)
\end{equation}
where the new variables are $\xi(x,t) = x + ct$ and $\eta(x,t) = x - ct$.
Note that by performing the change of variables to $(\xi,\eta)$ the wave
equation becomes $\partial^2 u /\partial\xi\partial\eta = 0$.

\begin{defn}[Cauchy problem for the one dimensional wave equation]
  \label{def:wave:cauchy}
  \[
    \begin{cases}
      u_{tt} - c^2 u_{xx} = 0, & (x, t) \in \mathbb{R} \times (0, \infty), \\
      u(x,0) = f(x), \\
      u_t(x,0) = g(x).
    \end{cases}
  \]
\end{defn}

To solve the wave equation's Cauchy problem we assume that the has the form of 
\eqref{eqn:wave:solution}, so the two conditions become $u(x,0) = p(x) + q(x)
= f(x)$ and $\partial_t u(x,0) = p'(x) + q'(x) = f'(x) = g(x)$. Solving for
$p(x)$ and $q(x)$ we get
\begin{align*}
  p(x) &= \frac{1}{2}f(x) - \frac{1}{2c}\int_0^x g(z) \, dz + a, \\
  q(x) &= f(x) - p(x) = \frac{1}{2}f(x) + \frac{1}{2c} \int_0^x g(z) \, dz -a.
\end{align*}
By combining the two and adding back the time dependence we obtain
d'Alambert's solution.
\begin{thm}[D'Alambert's solution]
  The solution to Cauchy problem of definition \ref{def:wave:cauchy} is given
  by
  \[
    u(x,t) = \frac{f(x-ct) + f(x+ct)}{2} 
      + \frac{1}{2c} \int_{x-ct}^{x+ct} g(z) \,dz.
  \]
\end{thm}


