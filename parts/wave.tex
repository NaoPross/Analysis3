% vim: ts=2 sw=2 et tw=78 spell:

\section{Wave Equation}

\subsection{Homogeneous Case}

The homogeneous one dimensional wave equation is a linear second order
hyperbolic PDE:
\[
  u_{tt} - c^2 u_{xx} = 0
  \qquad (x, t) \in \mathbb{R} \times (0, \infty),
\]
where $c$ is to be interpreted as being a ``velocity''. To get an algebraic
intuition for the solution we can rewrite it as
\[
  (\partial_t - c\partial_x)(\partial_t + c\partial_x) u = 0,
\]
from which we can recognize the terms as transport equations and conclude that
the solutione are a linear combination of forward or backwards traveling
waves. The general solution to this problem can thus be written as
\begin{equation} \label{eqn:wave:solution}
  u(x,t) = p(\xi) + q(\eta)
\end{equation}
where the new variables are $\xi(x,t) = x - ct$ and $\eta(x,t) = x + ct$.
Note that by performing the change of variables to $(\xi,\eta)$ the wave
equation becomes $\partial_\xi \partial_\eta u = 0$.

\begin{defn}[Cauchy problem for the one dimensional wave equation]
  \label{def:wave:cauchy}
  \[
    \begin{cases}
      u_{tt} - c^2 u_{xx} = 0, & (x, t) \in \mathbb{R} \times (0, \infty), \\
      u(x,0) = f(x), \\
      u_t(x,0) = g(x).
    \end{cases}
  \]
\end{defn}

To solve the wave equation's Cauchy problem we assume that the has the form of 
\eqref{eqn:wave:solution}, so the two conditions become $u(x,0) = p(x) + q(x)
= f(x)$ and $\partial_t u(x,0) = -c p'(x) + c q'(x) = g(x)$. Solving for
$p(x)$ and $q(x)$ we get
\begin{align*}
  p(x) &= \frac{1}{2}f(x) - \frac{1}{2c}\int_0^x g(z) \, dz + a, \\
  q(x) &= f(x) - p(x) = \frac{1}{2}f(x) + \frac{1}{2c} \int_0^x g(z) \, dz -a,
\end{align*}
where $a$ is an integration constant. By combining the two and adding back the
time dependence we obtain d'Alambert's solution.
\begin{thm}[D'Alambert's solution]
  The solution to Cauchy problem of definition \ref{def:wave:cauchy} is given
  by
  \[
    u(x,t) = \frac{f(x-ct) + f(x+ct)}{2} 
      + \frac{1}{2c} \int_{x-ct}^{x+ct} g(z) \,dz.
  \]
\end{thm}

\subsection{Domain of influence and Characteristic Lines}

By inspecting d'Alambert's solution we should note that the solution only
depends on a wedge shaped region in the $(x,t)$ plane that starts at
$(f(x),0)$ and expands in time. The initial datum is propagated only in this
region between $x-ct$ and $x+ct$ that is known as the \emph{domain of
influence}. The aforementioned lines are the \emph{characteristic lines} of
the wave equation.

\subsection{Inhomogeneous Case}

In a more general setting the wave equation may be inhomogeneous, i.e. the
Cauchy problem is as follows.

\begin{defn}[Cauchy problem for the one dimensional inhomogeneous wave equation]
  \label{def:wave:inhom-cauchy}
  \[
    \begin{cases}
      u_{tt} =  c^2 u_{xx} + F(x,t), 
        & (x, t) \in\mathbb{R} \times (0, \infty), \\
      u(x,0) = 0, \\
      u_t(x,0) = 0.
    \end{cases}
  \]
\end{defn}

The initial conditions are set to zero for an easier solution, and their
effect can be added back later using superposition. To solve this case we also
use the change of coordinates to $(\xi,\eta) = (x-ct, x+tc)$. By the chain
rule the wave equation becomes
\[
  \frac{\partial^2 u}{\partial \xi \partial \eta}
  = -\frac{1}{4c^2} F\left(\frac{\eta - \xi}{2c}, \frac{\eta + \xi}{2c}\right).
\]
Integrating and using the fact that $\partial_\xi u(\xi,\xi) = 0$ because of
the zero initial conditions results in a double integral. After a change of
variables and making use of the superposition to add this solution to
d'Alambert's solution we obtain the following result.

\begin{thm}[D'Alambert's solution for the inhomogeneous wave equation]
  The solution to the general Cauchy problem of the inhomogeneous wave
  equation is given by
  \begin{align*}
    u(x,t) = &\frac{f(x-ct) + f(x+ct)}{2} 
      + \frac{1}{2c} \int_{x-ct}^{x+ct} g(z) \,dz \\
      &\quad + \frac{1}{2c}
        \int_0^t \int_{x+c(t-\tau)}^{x+c(t+\tau)} F(y, \tau) \, dy d\tau.
  \end{align*}
\end{thm}

A important observation is that the (double) integral is over a triangular
region $\Delta = \{ (y,\tau) : x - c(t - \tau) \leq y \leq c + c(t - \tau), 0
\leq \tau \leq t\}$ that is known as the \emph{domain of depence} of the wave
equation, since only the values in $F(\Delta)$ affect the solution.

