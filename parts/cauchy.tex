% vim: ts=2 sw=2 et tw=78:
\section{PDE and Cauchy Problems}

\begin{defn}[Linear PDE]
  A PDE is \emph{linear} if it has the form
  \[
    a_0 u + \sum_{i} a_{1,i} u_{x_i}
    + \sum_{i,j} a_{2,i,j} u_{x_i x_j}
    + \cdots = f(\mathbf{x})
  \]
\end{defn}

\begin{defn}[Quasilinear PDE]
  A PDE is said to be \emph{quasilienar} if it is linear in its highet
  derivatives.
\end{defn}

\begin{defn}[Strong or classical solution]
  A solution to a PDE is said to beh \emph{strong} or classical if all
  of its derivatives exist and are continuous.
\end{defn}

\begin{defn}[2D Cauchy problem]
  \label{def:cauchy-2d}
  In 2 dimensions a \emph{Cauchy problem} has the form
  \begin{equation}
    \label{eqn:cauchy-2d}
    \begin{cases}
      a u_x  + bu_y = c \\
      u(x_0, y_0) = u_0(x_0, y_0)
    \end{cases}
  \end{equation}
  where $a, b, c$ may be functions of $x, y$ and $u$.
\end{defn}


