% vim: ts=2 sw=2 et tw=78 spell:
\section{Second Order PDEs}

A second order PDE of two variables has in the most general case the form
\[
  Au_{tt} + Bu_{tx} + Cu_{xx} + Du_{t} + Eu_x + Fu = G.
\]
To classify them we define the \emph{discriminant}
\[
  \Delta = B^2 - 4AC,
\]
which is reminiscent of the quadratic equation. Now, the PDE is said to be
\begin{itemize}
  \item \emph{hyperbolic} if $\Delta(x,t) > 0$,
  \item \emph{parabolic} if $\Delta(x,t) = 0$ and $A^2 + B^2 + C^2 \neq 0$,
  \item \emph{elliptic} if $\Delta(x,t) < 0$,
  \item \emph{singular} if $A = B = C = 0$.
\end{itemize}
It is important to note that since the parameters may be function of $(x,t)$,
so is $\Delta(x,t)$ and the PDE may be different depending on where we are in
the domain. Some well known examples: the wave equation is hyperbolic, the
heat equation is parabolic and the Laplace equation is elliptic.
